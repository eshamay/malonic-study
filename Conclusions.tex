\section {Conclusions}

The adsorption of small organic and chemically active molecules has gained great interest in recent years. Understanding the reactivities, orientations, adsorption pathways, and surface behaviors of small organics is of primary concern in building accurate atmospheric climate models, and in defining the many aqueous environments found on earth. Yet, the specific molecular nature of such organics, their surface geometries, orientations, and water interactions remain poorly understood. Although our comprehension of surface processes and interfacial chemistry is still in its infancy, we are beginning to gain new insights that are key to understanding environmentally important processes at aqueous surfaces.

Presented herein are results of KS-DFT MD simulations that focus on how a malonic acid behaves on a water surface, and the resulting orientations, geometries, and structures formed because of its interactions with interfacial water molecules. This computational study complements and expands on experimental studies from this laboratory that elucidated the molecular behavior of malonic acid.\cite{Blower2012} Furthermore, these computations build upon and enhance our understanding of malonic acid from the computational study on interfacial orientation and geometry of aqueous surface malonic acid molecules.

Our simulations show that the previously used classical interaction potential, based on point charges and polarizabilities, captures the orientation and hydration structure of the KS-DFT model. The complementary simulations studies using both interaction potentials show strong agreement for the surface-bound malonic acid behaviors. However, the classical model does not fully capture the resonance structures of the carboxylic acids, leading to minor discrepancies between the water structures that form to hydrate the acids.

Our analysis of intramolecular atomic distances resulted in the discovery of two dominant conformations of surface malonic acid: an intramolecularly hydrogen-bonded species, and an acid with more equivalent carboxylic acid ends that do not take part in internal bonding. The hydrogen bonded structure forms a ring-like internal geometry due to the folding of one carboxylic acid end towards the other, and subsequent hydrogen-bonding. The internal H-bond is strong and persistent throughout 40 ps of simulation, and warrants further experimental study to verify the existence of the species. The distribution of bondlengths in the surface malonic acid indicate that the two conformations of the acid have non-equivalent carboxylic acid moieties. In the hydrogen-bonded form, the bondlengths in one carboxylic acid end of the molecule are slightly longer or shorter than the other end. Delocalization of the atoms involved in the hydrogen bond causes elongation of the interacting bonds. The bondlengths and molecular geometry are accompanied with orientational and spectral changes brought on by the internal bonding of the molecule.

We have shown the results of an orientational analysis defining the overall molecular orientation of malonic acid on a water surface, and the configuration of the two carboxylic acids. The carbon backbone was found to lie slightly tilted from the plane of the interface. The hydrogen-bonded conformation has much more orientational freedom than its unbonded counterpart. Analysis of the dihedral angles of the carboxylic acids showed two distinct trends dependent on the specific molecular bonding conformation. The internally unbonded molecule's carboxylic acids orient very similarly to those found in our previous computational stud, with one aligning perpendicular to the other. However, the internally H-bonded acid has a very defined internal orientation, with one carbonyl aligned anti-parallel to the other due to a rigid backbone structure reinforced by the intramolecular bonding.

The orientation of the carbonyl bonds was analyzed to complete the picture of the molecular orientation of surface malonic acid. Two distinct carbonyl C=O behaviors were found to be preferred, with one C=O bond pointing more in the plane of the water surface, and the other pointing more perpendicularly, in towards the water bulk. This result agrees well with our previous classical simulation results, and also complements the conclusions from our VSFS experiments showing two preferred net orientations of the carbonyl C=O modes. As with the other computational results, the H-bonded form has greater orientational freedom overall for the carbonyl C=O bonds.

Lastly, to compare the KS-DFT vibrational modes with experiment, the local mode bond spectra were calculated for the carbonyl C=O bonds. The frequencies and spectral shape agree with what we found experimentally via VSFS of the surface-bound C=O modes. Although direct comparison of the spectra is not possible, the resulting frequency range reinforces the predictions of our simulations. Additionally, the C=O spectra of the internally H-bonded conformation show a splitting of the two C=O modes because of the drastically different environments surrounding each bond.

These studies build upon our computational and experimental research in this area, seeking to understand how small organic molecules behave while bound to, and reacting with an aqueous/air interface. Such invaluable knowledge is our key to understanding atmospheric aerosol and land water systems where small and reactive organic molecules bind to aqueous surfaces, and form platforms for interesting chemical processes. Further studies of these systems will aid us in better understanding water surfaces where molecular behavior can surprise us, and often defies our physical intuitions.
