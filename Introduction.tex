\section {Introduction}

Despite our recent experimental and theoretical achievements with simple organic acids in environmentally relevant systems, it is clear that our scientific understanding has far to go. The interfacial region of an aqueous solution is a turbulent and dynamic environment where the behaviors of even small organic molecules evade definition. How does the interfacial region alter behavior and strength of organic acid solutes? Do acid molecules interact with water at a surface as they do in bulk? What hydrate species and behavioral differences occur at an interface that are not found deeper in a liquid phase? Experiments addressing these types of questions give us valuable insight and information, but have not to date fully captured actual microscopic behaviors and events. Computationally, however, these systems can be more fully characterized. Coupling computational results with previous experimental work provides a much more complete picture of acid behavior throughout aqueous interfacial regions.

Organic acids are a particularly interesting candidates for studying aqueous acid behavior. Dicarboxylic acids are a pertinent class of hygroscopic, water-soluble, and atmospherically relevant bolaamphiphilic molecules, receiving much attention in recent years both experimentally,\cite{Peng2001,Kawamura1993,Kawamura1996,Kawamura1996a,Kawamura1999,Senpere1994,Senpere1996,Aggarwal2008,Hsieh2007,Hsieh2009,Pavuluri2010,Nieminen1992,Nahalovsk1970,Dam1983,Mahiuddin2008,Odum1996,Odum1997,Pandis1991,Zhang1992,Hoffmann1997,Hori2003,Bilde2003,Nilsson1998} and in theoretical computational studies.\cite{Darvas2010,Darvas2011,Dlugosz2004,Mohajeri2004,Krijn1988,Chen2000,Nieminen1992,Mahiuddin2008,Ma2011,Nilsson1998} They vary in size from the smaller oxalic acid to larger humic-like substances.\cite{Chebbi1996,Badger2005} Dicarboxylic acids make up an appreciable amount of the atmospheric organic particulate matter, and are implicated in the nucleating condensation of clouds.\cite{Darvas2010,Darvas2011,Cruz1997,Zobrist2006,Hori2003,Shantz2003} Because of their presence in the atmosphere from biogenic and industrial processes, in various types of particulates and aerosols, they are known to affect climate conditions and atmospheric chemistry.\cite{Kanakidou2005,Finlayson-Pitts2000,Seinfeld1998,Darvas2010,Darvas2011,Zobrist2006,Yan2008,Odum1996,Odum1997,Pandis1991,Zhang1992,Hoffmann1997,Hori2003,Kawamura1996a,Bilde2003,Shantz2003} 

Malonic acid, the second-smallest of the dicarboxylic acids, has been studied in binary and heterogeneous reactions, and in aerosols to develop cloud nucleation models.\cite{Giebl2002,Finlayson-Pitts2009} It has previously been studied experimentally with several recent publications attesting to its importance.\cite{Parsons2004,Braban2003,Hansen2004,Hyvarinen2006,Riipinen2007} Many computational theoretical works have also probed the nature of malonic acid in small cluster systems, at aqueous surfaces, and in gas phase.\cite{Nguyen2005,Merchan1984,Ma2011}

In this study we use ab initio molecular dynamics (MD) techniques to model and simulate the hydrating water structures that form around an interfacial malonic acid in water. The quantum MD technique described herein allows more realistic and accurate simulation than our previous classical MD study.\cite{Blower2012} In our prior work we determined net orientational behavior of malonic acid, coupled with experimental spectroscopic results to build a refined model of the acid's behavior at the air/\wat~interface. However, the classical interaction potential used needs to be tested against accurate quantum potentials to further verify the validity of the results obtained.

Quantum DFT MD simulation is the natural follow-up to the classical MD study as the interaction potential accurately reproduces hydration geometry around surface acid molecules. From the simulation data we examine in detail the specific bonding interactions that occur between surface waters and the carboxylic acid moieties of the malonic acid molecule, and look at the geometries and orientations of the hydrated acid molecule. Five concurrent simulations have been performed in this work, each of a malonic acid molecule bound to a water system surface. Each system was simulated at a temperature, solution concentration, and pH set to match the conditions of our most recent experimental studies.\cite{Blower2012} Our experiments showed that an aqueous malonic acid has a surface propensity in low-pH conditions. Although conclusions regarding the specific nature of those surface-bound hydrate complexes could only be inferred from experimental results, our previous and current computational simulations provide us with insights about the hydrated geometries of the acid molecules, and their orientational behavior.

We believe this computational study to be a necessary step in the development of models of malonic acid, and to continue building the picture of aqueous acid behavior. In this work we present a comparison between the classical and DFT interaction potentials to verify the validity of our previously used fully atomistic classical potential. We examine the internal geometry of surface malonic acid molecules, and discuss the behavioral implications of an interesting intramolecularly hydrogen-bonded species of malonic acid. These results complement several experimental and computational studies of the intramolecular interaction in several organic diacid conformers.\cite{Mohajeri2004,Dopieralski2011,Darvas2011,Nilsson1998,Chen2000,Eberson1959,Macoas2000,Macoas2000a,Merchan1984,Nguyen2005,Nieminen1992} We also show the orientational behavior of the aqueous surface acid molecule interacting with neighboring waters, specifying how the acid orients both with respect to the water surface, and internally by twisting about the carbon backbone bonds. Lastly, we analyze the vibrational behavior of the carbonyl modes of the acid to compare with, and complement our previous experimental results, and to further strengthen the link between our computational and experimental efforts.
