\section {Introduction}

With all of our recent experimental and theoretical progress, our scientific understanding of acidic solute behaviors at liquid water surfaces remains poorly developed. Much like the world's oceans appeared to early explorers centuries ago, the interfacial region of an air/water solution interface is a turbulent and dynamic boundary evading definition. How does the liquid interfacial region alter an acid's behavior and strength? Do acid molecules interact with water at a surface as they do in bulk? What hydrate species and behaviors occur at an interface that are not found deeper into a liquid phase? Experiments addressing these types of questions give us valuable insight and information, but have not to date fully captured such microscopic behaviors and events. Computationally, however, these systems can be fully characterized. Coupling computational results with previous experimental work provides a much more complete picture of acid behavior throughout aqueous interfacial regions.

Organic acids are a particularly interesting starting place for studying aqueous acid behavior because of their presence in the atmosphere,\cite{Kanakdou2005} and their role in aerosol sprays, known to affect climate conditions.\cite{Finlayson-Pitts2000,Seinfeld1998} Dicarboxylic acids are one such class of hygroscopic, water-soluble, and atmospherically relevant molecules.\cite{Peng2001,Kawamura1993,Kawamura1996,Kawamura1999,Senpere1994,Senpere1996,Aggarawal2008,Hsieh2007,Hsieh2009,Pavuluri2010} They vary in size from the smaller formic acids to larger humic-like substances.\cite{Chebbi1996,Badgerr2005} Dicarboxylic acids are implicated in the nucleating condensation of clouds.\cite{Cruz1997} Malonic acid, specifically, has been studied in binary and heterogeneous reactions and aerosols to develop cloud nucleation models.\cite{Giebl2002,Finlayson-Pitts2009} Malonic acid has previously been studied experimentally with several recent publications attesting to its importance.\cite{Parsons2004,Braban2003,Hansen2004,Hyvarinen2006,Riipinen2007} Many computational theoretical works have also probed the nature of malonic acid.\cite{Nguyen2005,Merchan1984,MoreReferences}

In this study we use ab initio molecular dynamics (MD) techniques to model and simulate the hydrating water structures that form around an interfacial malonic acid in water. The quantum MD technique described herein allows more realistic and accurate simulation than our previous classical MD study.\cite{Blower2012} In our previous work we determined net orientational behavior of malonic acid, coupled with experimental spectroscopic results to build a refined model of the acid's surface behavior. However, the classical interaction potential used needs to be further developed and tested against accurate quantum potentials to verify the validity of the results obtained.

Quantum MD simulation is the natural follow-up as the interaction potential accurately reproduces hydration geometry around surface acid molecules. From the simulation data we examine in detail the specific bonding interactions that occur between surface waters and the carboxylic acid moieties of the malonic acid molecule, and look at the geometries and orientations of the hydrated molecules. Five simulations have been performed in this work, each of a malonic acid molecule bound to a water system surface. Each system is simulated at 300K and at a solution concentration and pH set to match the conditions of our most recent experimental studies. Our experiments showed that an aqueous solution of malonic acid exhibits a surface-bound species, and two separate environments form in solvating the carbonyl bonds of the molecule. Although conclusions regarding the specific nature of those surface-bound hydrate complexes could only be inferred from experimental results, our previous and current computational simulations provide us with insights about the hydrated geometries of the acid molecules, and their orientational behavior.

We believe this to be a necessary step in the development of computational models of malonic acid, and to continue building the picture of aqueous acid behavior. In this work we present a comparison between the classical and DFT interaction potentials to verify the validity of our previously used fully atomistic classical potential. We examine the internal geometry of surface malonic acid molecules, and propose an interesting intramolecularly hydrogen-bonded species of acid. We also show the orientational behavior of the aqueous surface acid molecule interacting with neighboring waters, specifying how the acid orients both with respect to the water surface, and internally by twisting about the carbone backbone bonds. Lastly, we analyze the vibrational behavior of the carbonyl modes of the acid to compare with, and complement our previous experimental results, and to further strengthen the link between our computational and experimental efforts.
