\section {Computational Methods}

On-the-fly ab initio molecular dynamics simulations were performed with the QUICKSTEP package, which is an implementation of the Gaussian plane wave method using the Kohn-Sham formulation of density functional theory (KS-DFT).\cite{VandeVondele2005} The Kohn-Sham orbitals are expanded using a linear combination of atom-centered Gaussian-type orbital functions. The electronic charge density was described using an auxiliary basis set of plane waves. Energies and forces from on-the-fly simulation sampling of the Born-Oppenheimer surface were calculated for each MD step using the Gaussian DZVP basis set, the exchange-correlation functional of Becke, Lee, Yang, and Parr (BLYP),\cite{Lee1988} and the atomic pseudo-potentials of the Goedecker, Teter, and Hutter type.\cite{Goedecker1996} A simulation timestep of 1 fs was used, with a Nose-Hoover thermostat set at 300K. These computational parameters were verified to yield a reasonable description of bulk room temperature water when simulating a neat-water system, and in our previous computational studies with additional constituents.\cite{Shamay2007}

Five acid-water systems of unit cells sized 10x10x15\angs$^3$ were created as starting points for concurrent simulation.  Each unit cell was initially randomly packed with 34 water molecules, and 2 HCl molecules.  The system size was then expanded by an additional 10\angs~in the long cell dimension to final dimensions of 10x10x25\angs$^3$.  Periodic boundaries were then set on all axes to form an infinite aqueous slab configuration.  A single malonic acid molecule was then added onto the top of each water cell within 2\angs~of the topmost water molecule to simulate a surface-bound malonic acid at the point of adsorption.  System energies were then minimized through a geometry optimization procedure.  Subsequently, each system was equilibrated for 1 ps in canonical ensemble (NVT) conditions.  Using the equilibrated systems as starting points, each was simulated for a further 40 ps in the microcanonical ensemble (NVE), with coordinate snapshots recorded every 1 fs.  This simulation process resulted in 40,000 time steps of system trajectory for analysis in each of the five system replicas.
