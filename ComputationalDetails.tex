\section {Computational Methods}

On-the-fly ab initio molecular dynamics simulations were performed with the QUICKSTEP package, which is an implementation of the Gaussian plane wave method using the Kohn-Sham formulation of density functional theory (DFT).\cite{VandeVondele2005} The Kohn-Sham orbitals are expanded using a linear combination of atom-centered Gaussian-type orbital functions. The electronic charge density was described using an auxiliary basis set of plane waves. Energies and forces from on-the-fly simulation sampling of the Born-Oppenheimer surface were calculated for each MD step using the Gaussian DZVP basis set, the exchange-correlation functional of Becke, Lee, Yang, and Parr (BLYP),\cite{LEE1988} and the atomic pseudo-potentials of the Goedecker, Teter, and Hutter type.\cite{Goedecker1996} A simulation timestep of 1 fs was used, with a Nose-Hoover thermostat set at 300K. These computational parameters were verified to yield a reasonable description of bulk room temperature water when simulating a neat-water system, and in our previous computational studies.\cite{Shamay2007}

Five equilibrated acid-water systems of size 10x10x15\angs were used in the simulations. Each system was initially randomly packed with 34 water molecules, and 2 HCl molecules. The system size was then expanded by 10\angs in the long cell dimension for final dimensions of 10x10x25\angs. A single malonic acid molecule was then added onto the top of the water phase within 2\angs of the topmost water molecule. The system energy was minimized through a geometry optimization procedure. Subsequently, the system was equilibrated for 1 ns in canonical ensemble (NVT) conditions. Periodic boundaries were set on all axes to form an infinite slab configuration. The equilibrated systems were then simulated for a further 30 ps in the microcanonical ensemble (NVE), with trajectory snapshots recorded every 1 fs. The initial 1 ns equilibration trajectory was not included in the final analysis. This simulation process resulted in 30,000 time steps of system trajectory for analysis in each replica of the system.
 
